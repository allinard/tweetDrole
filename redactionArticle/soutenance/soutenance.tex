\documentclass{beamer}
%
% Choose how your presentation looks.
%
% For more themes, color themes and font themes, see:
% http://deic.uab.es/~iblanes/beamer_gallery/index_by_theme.html
%
\mode<presentation>
{
  \usetheme{Warsaw}      % or try Darmstadt, Madrid, Warsaw, ...
  \usecolortheme{default} % or try albatross, beaver, crane, ...
  \usefonttheme{default}  % or try serif, structurebold, ...
  \setbeamertemplate{navigation symbols}{}
  \setbeamertemplate{caption}[numbered]
} 

\usepackage[english]{babel}
\usepackage[utf8x]{inputenc}
\usepackage{booktabs}
\usepackage{multirow}


\title[Détection automatique des tweets humoristiques]{Est-ce que ce tweet est drôle ? \\ Détection automatique des tweets humoristiques}
\author{Adeline Granet, Alexis Linard}
\institute{Université de Nantes}
\date{18 avril 2014}

\begin{document}

\begin{frame}
  \titlepage
\end{frame}

% Uncomment these lines for an automatically generated outline.
%\begin{frame}{Outline}
%  \tableofcontents
%\end{frame}

%Introduction
\section{Introduction}
\subsection{Contexte}
\begin{frame}[allowframebreaks]{Définition}
	\begin{columns}
		\column{9cm}
		\begin{block}{Twitter}
			\begin{itemize}
				\item Un des leaders parmi les réseaux sociaux
				\item 645 millions d'utilisateurs dont 7.3 millions en France
				\item 58 millions de tweets par jour
			\end{itemize}
		\end{block}
		
		\column{2.5cm}
		\begin{figure}
			\includegraphics[width=2cm]{twitter.png}
		\end{figure}
	\end{columns}



\begin{block}{Tweet}
\begin{itemize}
\item Limité à 140 caractères
\item Auteurs (@)
\item Hashtags (\#)
\item Retweets, Retweeté, liens...
\item Contenu bruité
\end{itemize}
\end{block}

\end{frame}

\subsection{Problématique}
\begin{frame}{Problème du format}
	\begin{itemize}
		\item message subjectif
		\item message très court
		\item langage utilisé SMS, fautes de frappes
	\end{itemize}


\end{frame}



\begin{frame}{Exemple de tweets}
\begin{exampleblock}{Subjectif}
Quand je me moque des handicapés on me dit, mets toi à leurs places et quand je me met à leurs places on me mets une amende de 135 euros !
\end{exampleblock}
\begin{exampleblock}{Court}
Il court, il court le furet \#Contrepeterie
\end{exampleblock}
\begin{exampleblock}{Langage SMS}
mr6 bcp pr vs retweet lol!
\end{exampleblock}
\end{frame}

\subsection{Objectifs}
\begin{frame}{Objectifs}
Détection de l'humour

	
\begin{itemize}
\item Textes courts
\item Données bruitées
\item En français :
\begin{itemize}
\item Pas de ressources \textit{libres}
\item Pas d'étiqueteurs morpho-syntaxiques \textit{libres} 
\end{itemize}
\end{itemize}
\end{frame}






%table of contents
\begin{frame}
\frametitle{Sommaire} % Table of contents slide, comment this block out to remove it
\tableofcontents[hideallsubsections]  % Throughout your presentation, if you choose to use \section{} and \subsection{} commands, these will automatically be printed on this slide as an overview of your presentation
\end{frame}














%Etat de l'art
\section{Etat de l'art}

\subsection{Raz, 2012}
\begin{frame}[allowframebreaks]{\textit{Automatic humor classifcation on twitter}\\ de Raz, 2012 \cite{Raz12}}

\begin{itemize}
 \item méthode de classification de tweets humoristiques en anglais selon le type de l’humour;
\item algorithme semi-suprevisé
\item en entrée : des tweets, en sortie : des ensembles avec les même caractéristiques
\end{itemize}

\begin{block}{Caractéristiques lexicales}
\begin{itemize}
\item appartenance à des lexiques particuliers
\item présence des entités nommées
\item ambiguité
\end{itemize}
\end{block}

\vspace*{1.1cm}
\begin{block}{Caractéristiques morphologiques}
\begin{itemize}
\item analyse du temps des verbes
\item les mots existent (ou non)
\end{itemize}
\end{block}

\vspace*{1.1cm}
\begin{exampleblock}{Exemple : Phonologie}
\begin{itemize}
\item Leonard devint Sy
\end{itemize}
\end{exampleblock}

\vspace*{1.1cm}
\begin{block}{Style}
\begin{itemize}
\item présence de smiley
\item ponctuation particulière
\item hashtag
\end{itemize}
\end{block}

\begin{exampleblock}{Exemple : Style}
\begin{itemize}
\item On dit que le chien est le meilleur ami de l’homme mais les chiennes c’est pas mal non plus :)
\item Quelle est l’expression préférée d’un vampire ? Bon sang ! ! !
\end{itemize}
\end{exampleblock}


\begin{block}{Problème de cette méthode}
\begin{itemize}
\item necéssité d'avoir beaucoup de ressources : lexiques de mots drôles, d'homophones, mots vulgaires...
\item utilisation d'un site fermé aujourd'hui pour collecter les tweets 
\end{itemize}
\end{block}

\end{frame}


\subsection{Barbosa, 2010}
\begin{frame}[allowframebreaks]{\textit{Robust detection on twitter from biased and noisy data}\\ de Barbosa, 2010 \cite{Barbosa2010}}


Détection automatique de sentiments émis dans les tweets dont les caractéristiques sont :
\begin{itemize}
\item POStagging;
\item la polarité et la syntaxe spécifique du tweet comme les liens;
\item la ponctuation, les émoticônes, la casse des mots.
\end{itemize}
\end{frame}


\subsection{Autres}
\begin{frame}[allowframebreaks]{Autres}
\begin{block}{\textit{Evaluating humour features on web comments} de Reyes,2010 \cite{ReyesPRS10}}
Evaluation du modèle d'humour dans les commentaires de blog
\begin{itemize}
\item liste des termes exprimant des sentiments différents : les termes à caractère sexuel, à polarité négative, sémantiquement ambigus, reflètant des sentiments, et les émoticones et termes d’argot internet. 
\item évaluation sur un corpus de plus d’un millions de commentaires
\item résultats 60\% de précision
\end{itemize}
\end{block}

\begin{block}{\textit{Characterizing humour: An exploration of features in humorous texts} de Mihalcea,2007 \cite{MihalceaP07}}
\begin{itemize}
\item les faiblesses de l'homme : l’alcool, bière, ignorance, stupidité
\item le domaine professionnel, exemple "Le comble pour un dentiste, c’est d’habiter dans un palais."
\end{itemize}
\end{block}
\end{frame}


\section{Méthode}
\subsection{Méthodologie}
\begin{frame}{Méthodologie}

\begin{itemize}
  \item Création de corpus d'entraînement, ainsi que de test
  \item Identification des traits
  \item Entraînement avec Weka, et phase de test
  \item Identification des mesures intéressantes pour notre étude
  \item Amélioration avec scores de confiance
\end{itemize}


\end{frame}




\subsection{Les traits}
\begin{frame}{Les traits}

3 caractéristiques : 
\begin{itemize}
  \item Lexicales
  \item Stylistiques
  \item Contextuelles
\end{itemize}


\end{frame}






\begin{frame}[allowframebreaks]{Les traits}

\begin{block}{Caractéristiques lexicales}
Lexique de mots construit sur la base des mots racinisés du corpus

\end{block}


\begin{block}{Caractéristiques stylistiques} 
\begin{itemize}
\item La présence de hashtag :\\
Exemple :"\#humour" "\#contrepètrie"
\item La présence de smiley content ou pas content :\\
Exemple :"c'était pas moi ;)"
\item Le nombre de points d'exclamation: :\\
Exemple :"je suis calme!" vs " je suis calme !!!!!!!!!!" 
\end{itemize}
\end{block}


\begin{block}{Caractéristiques contextuelles}
\begin{itemize}
  \item Nombre de mots dans le tweet
  \item Nombre de retweets
  \item Longueur totale du tweet
  \item S'il s'agit d'un retweet
\end{itemize}
\end{block}


\end{frame}




\subsection{Méthodes de classification}
\begin{frame}{Environnement de travail}

\begin{block}{Weka}
Université de Waikato, Nouvelle-Zélande : suite populaire de logiciels d'apprentissage automatique parmi lesquels se trouvent des programmes réalisant de la classification.
\end{block}

\end{frame}



\begin{frame}{Les méthodes utilisées}

Les algorithmes de classification utilisés :
\begin{itemize}
  \item NaiveBayes
  \item J48
  \item MultilayerPerceptron
  \item DecisionStump
  \item RandomForest
\end{itemize}

\end{frame}





%Données
\section{Présentation des données}


\begin{frame}[allowframebreaks]{Corpora d'entraînement}
Peu de ressources libres en français dans les domaines du tweets et de l'humour
\begin{itemize}

\item Construction :\\ \textbf{1 corpus équilibré} et \textbf{1 corpus déséquilibré}
\item Outil : twitter4J
\item Choix des comptes "Drôles" : par mots clés \textit{@100\_blagues, @BlaguesCarambar, @BlagueJour}, par notoriété \textit{@VDM}
\item Choix des comptes "Pas Drôles" : politique \textit{@elysee}, journalistique \textit{@lemondefr} et commerciaux \textit{@m6, @nantesfr}
\end{itemize}




\begin{table}
\centering
\tiny
	\begin{tabular}{lrrrr}
	\toprule
	& \multicolumn{2}{c}{Equilibré}  & \multicolumn{2}{c}{Deséquilibré}\\
	\cmidrule(r){2-3} \cmidrule(r){4-5}

	& Tweets Drôle & Tweets Pas Drôles &  Tweets Drôle & Tweets Pas Drôles \\
	\midrule
	 ReTweets & 166 & 1019 & 166 & 1048 \\
	
	 ReTweetés & 2817 & 4009 & 0 & 0 \\
	
	Non ReTweetés & 2000 & 1273 & 4785 & 10541 \\
	
	Contrepètries & 200 & - & 200 & - \\
	Autodérision & 200 & - & 200 & - \\
	 \midrule
	Total & 4817 & 5282 & 4785 &  10541 \\
	\bottomrule
	\end{tabular}
\caption{Composition des corpus d'entraînement}
\label{corpus}
\end{table}

\end{frame}



\begin{frame}[allowframebreaks]{Corpus de test}


\begin{itemize}
  \item Validation croisée : problème de sur-apprentissage
  \item Choix des comptes : humouristes \textit{Gad Elmaleh,
Florence Foresti et Cyprien (un youtubeur)} 
  \item Tweets du quotidien et blagues 
\end{itemize}

\begin{block}{Phase d'annotation}
\begin{itemize}
\item Qui : 3 annotateurs
\item Tache : classer le tweet comme "Drôle" ou "Pas Drôle"
\item Combien : 250 tweets
\item Mesure : le coefficient $\kappa$
\end{itemize}
\end{block}

\vspace*{0.4cm}
Calcul centré sur les tweets annotés comme "Drôle" : 
\vspace*{0.2cm}
\begin{table}[!h]
\centering
	\begin{tabular}{lr}
	\toprule
	 Document & $\kappa$ \\
	\midrule
	  Fichier 1 & 0.978 \\
	  Fichier 2 &  0.967\\
	  Fichier 3 & 0.974  \\
	\bottomrule
	\end{tabular}
\caption{Résultat accords inter-annotateurs}
\label{anno}
\end{table}

\end{frame}






%Expérimentations
\section{Expérimentations}

\subsection{Paramètres expérimentaux}
\begin{frame}{Paramètres expérimentaux}
\begin{block}{Baseline}
On cherche à maximiser la précision :
\begin{itemize}
  \item Présence de smiley content/drôle vs pas content/pas drôle
  \item Présence de ponctuation en surnombre 
\end{itemize}
\end{block}


\begin{block}{Mesure d'évaluation}

\begin{itemize}
  \item La précision dans la détection des tweets drôles
  \item Les autres mesures sont peu importantes
\end{itemize}
\end{block}
\end{frame}




\subsection{Résultats}
\begin{frame}{Expérimentations}

\begin{table}[!h]
\centering
\tiny
	\begin{tabular}{llrrrr}
	\toprule

	& Indice de confiance & sans & 0.7 & 0.8 & 0.9 \\
	\midrule
    \multirow{4}{*}{Normal} & DecisionStump & 5.5\% & 0\% & 0\% & 0\% \\%0.7 confiance sans stem
	& J48 & 9.8\% & 10\% & 10\% & 11.5\% \\ %0.7
	& NaiveBayes & 11\% & 11.5\% & 12.3\% & 12.5\% \\ %0.7
	& RandomForest & 8.1\% & 10.4\% & 10.6\% & 25\%\\ %0.7
    
    \midrule
	\multirow{4}{*}{Racinisation}& DecisionStump & 5.5\% & 0\% & 0\% & 0\% \\%0.7 confiance AVEC stem
	& J48 & 7.5\% & 7.6\% & 7.6\% & 8.2\% \\ %0.7
	& NaiveBayes & 9.2\% & 10.7\% & 11.1\% & 11.1\% \\ %0.7
	& RandomForest & 8.3\% & 7.14\% & 7.4\% & 0\% \\ %0.7
    
    \midrule
	\multirow{4}{*}{Racinisation et Déséquilibré}& DecisionStump & 0\% & 0\% & 0\% & 0\% \\%0.7 confiance AVEC stem déséquilibré
	& J48 & 0\% & 0\% & 0\% & 0\% \\ %0.7
	& NaiveBayes & 0\% & 0\% & 0\% & 0\% \\ %0.7
	& RandomForest & 100\% & 100\%  & 100\% & 100\% \\ %0.7


	\bottomrule
	\end{tabular}
\caption{Résultats : précision sur la détection des tweets drôles, sans stemming, avec stemming, et avec stemming et corpus déséquilibré}
\label{precision}
\end{table}



\end{frame}








%Conclusion
\section{Conclusion}


\begin{frame}{Conclusion et Discussion}
\begin{block}{Amélioration possible}
\begin{itemize}
  \item L'étude du contexte de chaque tweet serait interessante
  \item Privilégier les mots les plus fréquents
\end{itemize}
\end{block}

\begin{block}{Limites de notre travail}
\begin{itemize}
  \item En anglais et sur des phrases complètes : résultats corrects
  \item Messages courts et bruités : peu de ressources
\end{itemize}
\end{block}
\vspace{0.5cm}
\hspace*{1.0cm}{\textbf{1 tweet humoristique sur 4 détecté reste satisfaisant}}
\end{frame}






\begin{frame}[allowframebreaks] %allow to expand references to multiple frames (slides)

\frametitle{References}

\scriptsize{\bibliographystyle{acm}}

\bibliography{biblio} %bibtex file name without .bib extension

\end{frame}


\end{document}
